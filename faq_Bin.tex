%%%%%%%%%%%%%%%%%%%%%%%%%%%%%%%%%%%%%%%%%
% Frequently Asked Questions
% LaTeX Template
% Version 1.0 (22/7/13)
%
% This template has been downloaded from:
% http://www.LaTeXTemplates.com
%
% Original author:
% Adam Glesser (adamglesser@gmail.com)
%
% License:
% CC BY-NC-SA 3.0 (http://creativecommons.org/licenses/by-nc-sa/3.0/)
%
%%%%%%%%%%%%%%%%%%%%%%%%%%%%%%%%%%%%%%%%%

\documentclass[11pt]{article}

\usepackage[margin=1in]{geometry} % Required to make the margins smaller to fit more content on each page
\usepackage[linkcolor=blue]{hyperref} % Required to create hyperlinks to questions from elsewhere in the document
\hypersetup{pdfborder={0 0 0}, colorlinks=true, urlcolor=blue} % Specify a color for hyperlinks
\usepackage{todonotes} % Required for the boxes that questions appear in
\usepackage{tocloft} % Required to give customize the table of contents to display questions
\usepackage{microtype} % Slightly tweak font spacing for aesthetics
\usepackage{palatino} % Use the Palatino font

\setlength\parindent{0pt} % Removes all indentation from paragraphs

% Create and define the list of questions
\newlistof{questions}{faq}{\large List of Frequently Asked Questions} % This creates a new table of contents-like environment that will output a file with extension .faq
\setlength\cftbeforefaqtitleskip{4em} % Adjusts the vertical space between the title and subtitle
\setlength\cftafterfaqtitleskip{1em} % Adjusts the vertical space between the subtitle and the first question
\setlength\cftparskip{.3em} % Adjusts the vertical space between questions in the list of questions

% Create the command used for questions
\newcommand{\question}[1] % This is what you will use to create a new question
{
\refstepcounter{questions} % Increases the questions counter, this can be referenced anywhere with \thequestions
\par\noindent % Creates a new unindent paragraph
\phantomsection % Needed for hyperref compatibility with the \addcontensline command
\addcontentsline{faq}{questions}{#1} % Adds the question to the list of questions
\todo[inline, color=green!40]{\textbf{#1}} % Uses the todonotes package to create a fancy box to put the question
\vspace{1em} % White space after the question before the start of the answer
}

% Uncomment the line below to get rid of the trailing dots in the table of contents
%\renewcommand{\cftdot}{}

% Uncomment the two lines below to get rid of the numbers in the table of contents
%\let\Contentsline\contentsline
%\renewcommand\contentsline[3]{\Contentsline{#1}{#2}{}}

\begin{document}

%----------------------------------------------------------------------------------------
%	TITLE AND LIST OF QUESTIONS
%----------------------------------------------------------------------------------------

\begin{center}
\Huge{\bf \emph{Notes: Frequently Asked Questions}} % Main title
\end{center}

\listofquestions % This prints the subtitle and a list of all of your questions

\newpage % Comment this if you would like your questions and answers to start immediately after table of questions

%----------------------------------------------------------------------------------------
%	QUESTIONS AND ANSWERS
%----------------------------------------------------------------------------------------

\question{How to add a package for latex?}\label{question-styinstall}
\begin{itemize}
  \item Get the name of the package.\\
  \item Search the package with google or CTAN.\\
  \item Download the package and unpack it, then handle the file with suffix .ins with latex.\\
  \item Move the whole package to tex directory. For Ubuntu: ~/texmf.
  \item Update the index of latex with command ``texhash''.\\
\end{itemize}

%------------------------------------------------

\question{How to enlarge the memory avaiable for imagJ?}\label{question-imagej}
One way:
\begin{verbatim}
su master
sudo vi imagej
change variable "default_mem" to the value you like.
\end{verbatim}

%------------------------------------------------

\question{How to use dpkg?}\label{question-dpkg}

(1) install *.deb package
\begin{verbatim}
dpkg -i *.deb
\end{verbatim}
(2) remove 
\begin{verbatim}
remove: dpkg -r pkg1 pkg2 ..
purge:  dpkg -P pkg1 pkg2 ..
\end{verbatim}
(3) inquiry package information
\begin{verbatim}
status:        dpkg -l [package]
file location: dpkg --listfiles {package}
detailed info: dpkg -s {package}
\end{verbatim}
(4) other useful skills for installing packages
\begin{verbatim}
apt-cache search: search package
apt-cache show: show the information of one package
apt-get install package --reinstall: reinstall package
\end{verbatim}

%------------------------------------------------

\question{In LAMMPS, what is the difference between \$\{x\} and v\_x?}\label{parase}

One example to explain: \\
\begin{verbatim}
    variable x equal temp
(1) variable y equal ${x}
(2) variable y equal v_x
\end{verbatim}
In the first command, if temp equals 10 now, then y alway equals 10. And in the second style, the command means $y=x=
temp$, y will change according to x or temp.\\
PS: when use \$x in variable command, quotes should not be added.
One example to get the current lz:
\begin{verbatim}
variable tmp equal lz
variable lz0 equal ${tmp}
\end{verbatim}
By the way, I want to note the way to add an increment to a variable
\begin{verbatim}
variable a equal 2
variable tmp euqal ``v_a + 1''
variable a equal ${tmp}
\end{verbatim}

%------------------------------------------------

\question{Why do we use ''dump\_modify 1 sort id'' when to get the atomic strain?}\label{atomic-strain}

Atomeye stores the atom vector in the order atom data appears in cfg file. If that command is not used, the atom order
output in the cfg file is not always the same. So, with the command, atomeye could store the atom data in the same
position in atom vector so as to get the atomic strain calculated correct.
%------------------------------------------------

\question{Why is backup so important?}\label{backup}

Backup is very very very very very \textbf{important} which you will find a true principle when you lose something. And when you try to use ''rm
-rf *'', think it over before you do it.\\
PS: However, it is possible to recover the files you deleted with ''rm''. But it is not an easy thing.

%------------------------------------------------

\question{How to simpely use tar?}\label{tar}

-c: create\\
-r: append file\\
-v: output info when handle files\\
-f: use archive file or device ARCHIVE\\
-t: list the contents of an archive\\
-x: extract files from an archive\\
-z: --gzip, --gunzip --ungzip\\
\begin{verbatim}
compress:      tar [-]czvf this.tar.gz --exclude=PATTERN files
decompress:    tar [-]xzvf this.tar.gz
list contents: tar [-]tvf  this.tar.gz
\end{verbatim}

%------------------------------------------------

\question{How to delete all the files (same name or pattern) in all directories below the current one?}\label{delete}

One possible way:
\begin{verbatim}
EXAMPLE: 
To remove all PNG images
find . -name '*.png' -print0 | xargs -0 rm
\end{verbatim}

%------------------------------------------------

\question{Is it possible to define the default scale in calculator bc?}\label{bc}

Here, I find a comprise. To use \textbf{bc -l} which will set scale = 20. In this way, maybe an alias could be set to
use bc if large scale is wanted in most situation. If you just want to use bc for one calculation, the alias below may
be useful to you.
\begin{verbatim}
alias mymath='bc -l <<<'
Usage: mymath '290/109'
\end{verbatim}
Tips: If you want calculate one formula with integer, it could be done like below:
\begin{verbatim}
echo $((2+3))
\end{verbatim}
which may be used in your bash script.\\
WARNING: bc doesn't support scientific notation. However, they can be translated in a format bc can handle by using sed.
By using
\begin{verbatim}
value=`echo ${value} | sed -e 's/[eE]+*/\*10\^/'`
\end{verbatim}
Thus, a bash script may be useful,
\begin{verbatim}
#!/bin/sh
value=`\varepsiloncho $1 | sed -e 's/[eE]+*/\*10\^/g'`
value=`\varepsiloncho $value | bc -l`
echo $value
\end{verbatim}
%------------------------------------------------

\question{How to use sed to replace one line with defined string?}\label{sed}

\begin{verbatim}
sed -i -e '5c\1000' file
\end{verbatim}
Some other usage:\\
Print selected lines:
\begin{verbatim}
sed -n '10,12p;13q' file
\end{verbatim}
Insert line:
\begin{verbatim}
sed -i '1 i Hello' test.dat
\end{verbatim}
Delete line:
\begin{verbatim}
sed -i '2d' test.dat
sed -i '/fish/d' test.dat
\end{verbatim}
Replace string:
\begin{verbatim}
sed -i 's/foo/test/g' filename
\end{verbatim}

%------------------------------------------------

\question{How to input some uncommon char in xmgrace?}\label{xmgrace}

In the input box, type \textbf{''ctrl+E''} may help you.
Angstrom 
\begin{verbatim}
\cE\C
\end{verbatim}
%------------------------------------------------

\question{How to count all the files in current directory?}\label{file-count}

The command below may be useful:
\begin{verbatim}
ls -l | grep '^-' | wc -l
\end{verbatim}
%------------------------------------------------

\question{How to use vim-latex? vim-latex notes}\label{vim-latex}
\begin{itemize}
\item type in \verb+\ref{fig:+ and press Ctrl-n you will automatically cycle through
 all the figure labels. 
\end{itemize}
PS: The vim-latex reference card makes a better summary. So I stop making notes.


%------------------------------------------------

\question{How to encrypt a file in linux?}\label{encrypt-file}

% cSpell:disable
The simplest way:
\begin{verbatim}
vim -x myfile
\end{verbatim}
Use openssl:
\begin{verbatim}
encrypt:
openssl enc -aes-128-ecb -e -in myfile -out myfile-aes
decrypt:
openssl enc -aes-128-ecb -d -in myfile-aes -out myfile
\end{verbatim}
Use GnuPG: man gpg
%------------------------------------------------
% cSpell:enable

\question{How to install Chinese surport in TEX-LIVE}\label{Chinese}

% cSpell: ignore texlive ctex fontset zhmetrics zhwinfonts
\begin{enumerate}
\item Find a texlive 2013 image to download.
\item \verb+sudo mount -o loop texlive-image /mnt+
\item Use command \verb+sudo apt-get install perl-tk+ to install texlive with GUI
\item Install: \verb+sudo ./install-tl -gui+
\item \verb+export PATH=/usr/local/texlive/2013/bin/x86_64-linux:$PATH+
\item Install Chinese fonts.
\item Modify Chines config file in /usr/local/texlive/2013/texmf-dist/tex/latex/ctex/fontset \&
 texmf-dist/tex/generic/zhmetrics/zhwinfonts.tex
\item It is recommended to use xelatex to compile your tex file.
\item other useful command: \verb+fc-cache    fc-list :lang=zh+
\end{enumerate}

%------------------------------------------------

\question{How to simply use sort?}\label{sort}

One example:
sort the second column of my.dat in ascending order.
\begin{verbatim}
sort -n -k 2 my.dat -o my.dat
\end{verbatim}

%------------------------------------------------

\question{How to make a comment in awk script}\label{awk-script}

The content behind \# will be commented.

One more tip in awk, the variable will be initialized as NULL string or zero, depending on the environment. It is due to
the dynamic type of variables in awk.

%------------------------------------------------

\question{How to output the data in gnuplot?}\label{gnuplot-data}

\begin{verbatim}
set table 'filename.dat'
plot 'tmp.dat' u 1:2 w p
unset table
\end{verbatim}
%------------------------------------------------

\question{How paste code in vim without clutters?}\label{vim:paste}

% cSpell: ignore nopaste vimrc pastetoggle
One useful set when pasting code:
\begin{verbatim}
set paste
set nopaste
\end{verbatim}
Another way:
\begin{verbatim}
Add to ~/.vimrc:
set pastetoggle=<F9>
\end{verbatim}
When you are in the insert mode, pressing $<F9>$ will toggle paste mode.
%------------------------------------------------

\question{How to output the sort result to the origin file?}\label{sort:output}

One example:
\begin{verbatim}
sort -n -k 1 -o data.file data.file 
\end{verbatim}
%------------------------------------------------

\question{How to plot 3D vector field with gnuplot?}\label{gnuplot:3d vector}

% cSpell: ignore splot Xcrysten
One example:
\begin{verbatim}
splot ``data.file'' w vectors
\end{verbatim}
You can also plot it with atomeye or Xcrysten, and atomeye works better in big system.\\
Enhancement to atomeye: http://www.jrkermode.co.uk/atomeye.html

%------------------------------------------------

\question{How to make the path name short on the console?}\label{console:short}

% cSpell: ignore textbackslash
Edit the ~/.bashrc, and on the line that starts PS1= change the lowercase 
``\textbackslash w'' into an upper case ``\textbackslash W''.

%------------------------------------------------

\question{How to get the graphic card info?}\label{system:graphic card}
% cSpell: ignore lspci
One command:
\begin{verbatim}
lspci
\end{verbatim}
Comments: List Peripheral Component Interconnect?
%----------------------------------------------------------------------------------------

\question{How to grep files recursively?}\label{grep:files}

Only include the first child directory:
\begin{verbatim}
grep 'Loop time' */*
\end{verbatim}
Find all files recursively:
\begin{verbatim}
find . -name '*' -print0 | xargs -0 grep 'Loop time'  
\end{verbatim}
%----------------------------------------------------------------------------------------

\question{How to extract images from pdf document in ubuntu/linux?}\label{pdfimage}

% cSpell: ignore pDfimages 
Pdfimages may help you. Some example:
\begin{verbatim}
pdfimages -j -p -f 2 -l 3 one.pdf name-prefix
\end{verbatim}
%----------------------------------------------------------------------------------------

\question{How to use rsync to update remote files?}\label{rsync}

% cSpell: ignore rsync xubin 
\begin{verbatim}
rsync -avz -e 'ssh -p 2222' xubin@manager:PATH/ local_path
\end{verbatim}
Pay attention to the ``/'' after path
%----------------------------------------------------------------------------------------

\question{How to plot data set in gnuplot?}\label{gnuplot-dataset}
Just add i option to choose data set. For example,
\begin{verbatim}
plot ``data_file'' i 0 u 2:4 w lp
\end{verbatim}

%----------------------------------------------------------------------------------------

\question{Where to find the host file?}\label{hostfile}
For linux: /etc/hosts\\ 
For win7: C:/windows/system32/driver/etc/hosts\\

%----------------------------------------------------------------------------------------

\question{How to update svn to a new computer?}\label{svn-co}
For example,
\begin{verbatim}
svn checkout http://artn.googlecode.com/svn/trunk/
\end{verbatim}

%----------------------------------------------------------------------------------------

\question{Some tips for using command top in linux.}\label{top}
Use 'h' to see the help info.\\
One useful command '1' to see every cpu info. 'c' to get colored
%----------------------------------------------------------------------------------------

\question{How to setup default mod for created files under linux?}\label{umask}
% cspell: ignore umask
umask helps you.

%----------------------------------------------------------------------------------------

\question{How to read both file and stdinput to an process? }\label{}

\begin{verbatim}
cat file - | program
\end{verbatim}

%----------------------------------------------------------------------------------------

\question{How to change apache Document root in Ubuntu 14.04 LTS?}\label{}
First,
\begin{verbatim}
sudo vi /etc/apache2/sites-available/000-default.conf
\end{verbatim}
Find
\begin{verbatim}
DocumentRoot /var/www/html
\end{verbatim}
Change the directory.
\begin{verbatim}
sudo vi /etc/apache2/apache2.conf
\end{verbatim}
Find 
\begin{verbatim}
<Directory /var/www/html/>
Options Indexes FollowSymLinks
AllowOverride None
Require all granted
</Directory>
\end{verbatim}
Change the directory. Then, restart apache.
\begin{verbatim}
sudo service apache2 restart
\end{verbatim}

%----------------------------------------------------------------------------------------

\question{How to install Pydio in Ubuntu?}\label{pydio}

% cSpell: ignore mcrypt pydio chown libapr1 libaprutil1
First, install mysql and php\ldots
\begin{verbatim}
sudo apt-get install apache2 php5 php5-cli php5-gd php5-mcrypt libapr1 libaprutil1 ssl-cert
\end{verbatim}
Download pydio, and mv it to /var/www. Then, 
\begin{verbatim}
chown  www-data:www-data -R /var/www/pydio
\end{verbatim}
Create database in mysql,
\begin{verbatim}
mysql -u root -p
CREATE DATABASE pydio
CREATE USER ” pydio ‘@’ localhost ‘ IDENTIFIED BY ‘ password ‘
GRANT ALL PRIVILEGES ON * TO pydio ” pydio ” @ “localhost”
\end{verbatim}
Change Apache AllowOverride. Find /etc/apache2/apache2.conf to change the flag to ``all''\\
Change max file size upload. Find php.ini
\begin{verbatim}
upload_max_filesize = ``20M''
post_max_size = ``25M''
\end{verbatim}
Configure your pydio through web setup wizard.

%----------------------------------------------------------------------------------------

\question{How to restart apache?}\label{restart-apache}
% cspell: ignore apachectl
With this easy command,
\begin{verbatim}
apachectl -k restart
\end{verbatim}

%----------------------------------------------------------------------------------------

\question{What are some funny Linux commands?}\label{linux-funny}
% cspell: ignore blinkenlights
For one,
\begin{verbatim}
telnet towel.blinkenlights.nl
\end{verbatim}


%----------------------------------------------------------------------------------------

\question{How to judge if a file or a directory exist in current directory in bash?}\label{bash-judge}

Examples,
\begin{verbatim}
[ -f file ] && echo ``file exist.''
[ -d directory ] && echo ``directory exist''
\end{verbatim}

%----------------------------------------------------------------------------------------

\question{Why is the total potential energy of L-J system  very small? }\label{}
Note that for LJ units, the default mode of thermodynamic output via the thermo\_style command is to normalize energies by
the number of atoms, i.e. energy/atom. This can be changed via the thermo\_modify norm command.

%----------------------------------------------------------------------------------------

\question{How to do spell check in VIM?}\label{}
For versin >= 7, we can turn the spell check on use the following setting:
\begin{verbatim}
:set spell spelllang=en_us
\end{verbatim}
Updates: A better way,
\begin{verbatim}
map <F4> :setlocal spell! spelllang=en_us<CR>
\end{verbatim}
To move o a misspelled word, use ]s and [s. The ]s command will move the cursor to the next misspelled word, the [s
  command will move the cursor back through the buffer to previous misspelled words.\\
  Once the cursor is on the word, use z=, and Vim will suggest a list of alternatives that it thinks may be correct.\\


%----------------------------------------------------------------------------------------

\question{How to debug shell script?}\label{}

Run shell script with option '-x'.
%----------------------------------------------------------------------------------------

\question{How to merge pdf files?}\label{}
\begin{verbatim}
pdftk file1.pdf file2.pdf cat output mergedfile.pdf
\end{verbatim}

%----------------------------------------------------------------------------------------

\question{How to use git?}\label{}

An useful guid (url: http://rogerdudler.github.io/git-guide/).
I may write a summary if I have spare time.
Easy guide:
First, you may do a bunch of changes to local files. Then,
\begin{verbatim}
git add min_artn.cpp
git commit
git push origin master
\end{verbatim}
Finished.

%----------------------------------------------------------------------------------------

\question{How to convert pdf to jpg?}\label{}

It is easy to use convert command in linux to convert file type. Example:
\begin{verbatim}
convert -density 300 file.pdf file.jpg
\end{verbatim}

%----------------------------------------------------------------------------------------

\question{How to write differential operator in latex}\label{}
It is easy to deal with this use commath package with command
\begin{verbatim}
\usepackage(commath}
\int \dif x
\end{verbatim}

%----------------------------------------------------------------------------------------

\question{How to let LaTex just put picture on the origin position?}\label{}
Use float package, and position flag use H.
\begin{verbatim}
\usepackage{float}
\begin{figure}[H]
\end{verbatim}

%----------------------------------------------------------------------------------------

\question{How to add daily jobs in ubuntu?}\label{}

Use crontab to do it,
\begin{verbatim}
crontab -e
\end{verbatim}

%----------------------------------------------------------------------------------------

\question{How to solve the Chinese chaos in gftp?}\label{}
\begin{verbatim}
vi ./gftp/gftprc
remote_charsets=cp936,gb18303,gbk,gb2312,utf8,gb18030,euc-tw,zh_CN
\end{verbatim}

%----------------------------------------------------------------------------------------

\question{How to make ipython show the history in previous session?}\label{}
\begin{verbatim}
%history ~1/
\end{verbatim}

%----------------------------------------------------------------------------------------

\question{How to clean DNS-cache in Ubuntu?}\label{}
\begin{verbatim}
sudo /etc/init.d/networking restart
\end{verbatim}

%----------------------------------------------------------------------------------------

\question{How to print with colors in terminal using python?}\label{}
Several approaches:
\begin{itemize}
  \item 
    \begin{verbatim}
class bcolors:
    HEADER = '\033[95m'
    OKBLUE = '\033[94m'
    OKGREEN = '\033[92m'
    WARNING = '\033[93m'
    FAIL = '\033[91m'
    ENDC = '\033[0m'
    BOLD = '\033[1m'
    UNDERLINE = '\033[4m'
print bcolors.WARNING + "Warning: No active frommets remain. Continue?" 
      + bcolors.ENDC    
    \end{verbatim}
  \item
    \begin{verbatim}
from termcolor import colored
print colored('hello', 'red'), colored('world', 'green')
    \end{verbatim}
   Or module colorrama for all cross-platform supporting.
  \item
    \begin{verbatim}
CSI=``\x1B[''
  reset=CSI+``m''
  print CSI+``31;40m'' + ``Colored Text'' + CSI + ``0m''
    \end{verbatim}
    For more info see http://en.wikipedia.org/wiki/ANSI\_escape\_code
\end{itemize}


%----------------------------------------------------------------------------------------

\question{How to unfold all folds in vim?}\label{}
\begin{verbatim}
zR
\end{verbatim}
In order to adjust the initial fold level, play around with the foldlevel, e.g.
\begin{verbatim}
:set foldlevel=1
\end{verbatim}


%----------------------------------------------------------------------------------------

\question{How to use Vim spell check?}\label{}
Active:
\begin{verbatim}
:setlocal spell spelllang=en_us `` For coding
:set spell spelllang=en_us '' For text editing
map <F4> :setlocal spell! spelllang=en_us<CR>
\end{verbatim}
Using spellchecking:\\
Move to a misspelled word, use ]s and [s. The ]s command will move the cursor to the next misspelled word, the [s
  command will move the cursor back through the buffer to previous misspelled words.
  Once the cursor is on the word, use z=, and Vim will suggest a list of alternatives that it thinks may be correct. 

%----------------------------------------------------------------------------------------

\question{Some tips in LAMMPS}\label{}
\begin{itemize}
  \item Sometimes, fix box/relax may cause a slow minimization porcess.
  \item Some fixes like langevin may change the force output for each atom. One should pay attention when trying to get
    atom force.
  \item One should be careful when using lattice orient to create box atoms, as it sometimes failed to create periodic
    boudary box. Or one should try to use block 4*N to avoid the bad situation.
    One example: fcc (111-z) 
    \begin{verbatim}
lattice         fcc  $a orient x 1 1 -2 orient y -1 1 0 orient z 1 1 1
region          box block 0 10 0 10 0 12
# change to
# region          box block 0 12 0 12 0 24
    \end{verbatim}
\end{itemize}

%----------------------------------------------------------------------------------------

\question{How to distribute user info in NIS server?}\label{}
\begin{verbatim}
sudo make -C /var/yp
\end{verbatim}
%----------------------------------------------------------------------------------------

\question{How to solve chaos in gedit chinese?}\label{}
\begin{verbatim}
gsettings set org.gnome.gedit.preferences.encodings auto-detected ``['GB18030', 'UTF-8', 'CURRENT', 'ISO-8859-15', 'UTF-16']''
\end{verbatim}
%----------------------------------------------------------------------------------------
%----------------------------------------------------------------------------------------

\question{How to easily use sympy?}\label{}
One example:
\begin{verbatim}
from sympy import *
x = Symbol('x')
k = Symbol('k')
diff(kx,x)
\end{verbatim}

\question{How to simply use GDB?}\label{}
Set args:
\begin{verbatim}
set args -in in.test
\end{verbatim}
print array:
\begin{verbatim}
print *a@10
\end{verbatim}
Other command:
\begin{verbatim}
run
continue (c)
watch
print (p)
display
b file:linenum
\end{verbatim}
rerun: run
%----------------------------------------------------------------------------------------
\question{How to find files in linux?}\label{}
\begin{itemize}
  \item[find]
    search for files in a directory hierarchy.\\
    Example: 
    \begin{verbatim}
    find . -name 'test*'
    \end{verbatim}
  \item[locate]
    locate reads one or more databases prepared by updatedb(8) and writes file names matching at least one of the
    PATTERNs to standard output, one per line. To find the files recently created, you should frist update database with
    command ``updatedb''
  \item[whereis] locate the binary, source, and manual page files for a command.
  \item[which] locate a command.
  \item[type]
\end{itemize}

%----------------------------------------------------------------------------------------
\question{How to write good paper?}\label{}
\begin{itemize}
  \item Read and Revise.
\end{itemize}

%----------------------------------------------------------------------------------------
\question{What is the difference in citing/referencing with or without tilde?}\label{}
The tilde ~ is an unbreakable space, i.e. the line will never be broken at this position. If you write
Table\verb+~\ref{\ldots}+ the table number generated by \verb|\ref| will always be on the same line as Table, which is the preferable
formatting. Having ``Table'' at the end of a line and then ``1'' at the beginning of the next simply looks bad.

The tilde is also used in names if they include a title, like \verb|Dr.~Faust|, which will also ensure that the ``Dr.'' and the
name is not broken between two lines, and also ensures that the \verb+.+ is not taken as a full-stop, which usually produces a
larger space after it.

%----------------------------------------------------------------------------------------
\question{HowTo: Linux / UNIX List Just Directories Or Directory Names}\label{}
\begin{verbatim}
ls -l | grep '^d'
ls -d */
\end{verbatim}


%----------------------------------------------------------------------------------------
\question{How to Reset the Numbering of Tables and Figures in an Appendix in latex?}\label{}
By default (at least in Koma-Script article class), Tables and Figures included in an Appendix will be numbered
continuously with the Tables and Figures in the main text. In order to have the ones in the Appendix numbered in a
special way (e.g., Figure A.1 for Appendix A), the following TeX hack can be used to reset and format the numbering.
Insert the following in ERT somewhere inside the Appendix before any floats (but in a standard paragraph, not in the
Appendix section title):
\begin{verbatim}
\setcounter{figure}{0} \renewcommand{\thefigure}{A.\arabic{figure}}
\end{verbatim}
To do the same for Tables, just change the string ``figure'' to ``table'' in all three places.
This changes the entries in the TOC correctly as well.
Ref: https://wiki.lyx.org/Tips/ResetNumberingOfTablesAndFiguresInAnAppendix

%----------------------------------------------------------------------------------------
\question{How to avoid font missing in latex figure from xmgrace?}\label{}
In the print setup in xmgrace, avoid the option using device font.

%----------------------------------------------------------------------------------------
\question{SShd safety strategy}\label{}
\begin{verbatim}
  /etc/ssh/sshd_config
\end{verbatim}
\begin{itemize}
  \item \verb+PermitRootLogin  no+
  \item \verb+PermitEmptyPasswords  no+
  \item \verb+Port 2222+
  \item \verb+ListenAddress 192.168.1.5+
  \item \verb+PasswordAuthentication no+
  \item change /etc/hosts.allow  
    \begin{verbatim}
    sshd:192.168.1.5:allow
    \end{verbatim}
    change /etc/hosts.deny
    \begin{verbatim}
     sshd:ALL
    \end{verbatim}
\end{itemize}


%----------------------------------------------------------------------------------------
\question{How to know the live host (ip, mac, hostname) in local network?}\label{}
\begin{verbatim}
nmap -sP 192.168.1.*
cat /proc/net/arp
\end{verbatim}


%----------------------------------------------------------------------------------------
\question{How to debug python script?}\label{}
\begin{verbatim}
python -m pdb myscript.py
\end{verbatim}

%----------------------------------------------------------------------------------------
\question{How to install and use shadowsockets?}\label{}
If you want to see beyond the great fire wall (GFW), the ``shadowsockets'' will be a powerful
tool.\\
\emph{Deploy a shadowsockets server.}
VPS: e.g. wwww.vultr.com\\
Setup the server:
\begin{itemize}
  \item Install neccessary packages
    \begin{verbatim}
apt-get update
apt-get install build-essential python-pip python-m2crypto python-dev python-gevent supervisor
pip install shadowsocks
    \end{verbatim}
  \item setup the shadowsocks
    \begin{verbatim}
vi /etc/shadowsocks.json
{
    "server":"45.32.15.178",
    "port_password":{
        "8388":"mima1",
        "8389":"mima2"
    },
    "timeout":600,
    "method":"aes-256-cfb"
}
EOF
\end{verbatim}

  \item setup supervisor
    \begin{verbatim}
vi /etc/supervisor/conf.d/shadowsocks.conf
[program:shadowsocks]
command=ssserver -c /etc/shadowsocks.json
autorestart=true
user= nobody
EOF
service supervisor start
supervisorctl reload
    \end{verbatim}
  \item misc
    \begin{verbatim}
vi /etc/security/limits.conf
Append:
* hard nofile 51200
* soft nofile 51200
EOF
vi /etc/profile
Append:
ulimit -n 51200
EOF
vi /etc/sysctl.d/local.conf
# max open files
fs.file-max = 51200
# max read buffer
net.core.rmem_max = 67108864
# max write buffer
net.core.wmem_max = 67108864
# default read buffer
net.core.rmem_default = 65536
# default write buffer
net.core.wmem_default = 65536
# max processor input queue
net.core.netdev_max_backlog = 4096
# max backlog
net.core.somaxconn = 4096

# resist SYN flood attacks
net.ipv4.tcp_syncookies = 1
# reuse timewait sockets when safe
net.ipv4.tcp_tw_reuse = 1
# turn off fast timewait sockets recycling
net.ipv4.tcp_tw_recycle = 0
# short FIN timeout
net.ipv4.tcp_fin_timeout = 30
# short keepalive time
net.ipv4.tcp_keepalive_time = 1200
# outbound port range
net.ipv4.ip_local_port_range = 10000 65000
# max SYN backlog
net.ipv4.tcp_max_syn_backlog = 4096
# max timewait sockets held by system simultaneously
net.ipv4.tcp_max_tw_buckets = 5000
# turn on TCP Fast Open on both client and server side
net.ipv4.tcp_fastopen = 3
# TCP receive buffer
net.ipv4.tcp_rmem = 4096 87380 67108864
# TCP write buffer
net.ipv4.tcp_wmem = 4096 65536 67108864
# turn on path MTU discovery
net.ipv4.tcp_mtu_probing = 1

# for high-latency network
# net.ipv4.tcp_congestion_control = hybla

# for low-latency network, use cubic instead
# net.ipv4.tcp_congestion_control = cubic
EOF
sysctl --system
reboot
    \end{verbatim}
\end{itemize}
Setup the client:\\
\begin{verbatim}
pip install shadowsocks
sudo vi /etc/shadowsocks.json
{
  "server":"my_server_ip",
  "local_address": "127.0.0.1",
  "local_port":1080,
  "server_port":my_server_port,
  "password":"my_password",
  "timeout":300,
  "method":"aes-256-cfb"
}
EOF
system network proxy: Automatic and need the file generated below.
pip install genpac
mkdir ~/shadowsocks
cd shadowsocks
genpac --proxy="SOCKS5 127.0.0.1:1080" --gfwlist-proxy="SOCKS5 127.0.0.1:1080" -o autoproxy.pac --gfwlist-url="https://raw.githubusercontent.com/gfwlist/gfwlist/master/gfwlist.txt"
EOF
\end{verbatim}
When using chrom, sometimes you may need SwitchyOmega.

%----------------------------------------------------------------------------------------
\question{How to make terminal use socks5 proxy?}\label{}
\begin{verbatim}
# start proxy
alias start_ss='export ALL_PROXY=socks5://127.0.0.1:1080'
# end proxy
alias end_ss='unset ALL_PROXY'
\end{verbatim}

%----------------------------------------------------------------------------------------
\question{How to assign output of os.system to a variable and prevent it from being displayed on the screen?}\label{}
\begin{verbatim}
import subprocess
output = subprocess.check_output(``cat /etc/services'', shell=True)
\end{verbatim}


%----------------------------------------------------------------------------------------
\question{How to reformat long width paragraph in vim to short width paragraph?}\label{}
\begin{verbatim}
gqip
\end{verbatim}

%----------------------------------------------------------------------------------------
\question{Tips for customizing Visual Studio Code?}\label{}
Useful extensions:
\begin{itemize}
  \item Bracket Pair Colorizer 2
  \item C/C++
  \item Code Spell Checker
  \item Git graphic
  \item Git History diff
  \item LaTex Workshop
  \item python
  \item Vim
\end{itemize}
%----------------------------------------------------------------------------------------

\question{How to find ebooks online?}\label{}
\begin{itemize}
  \item \url{https://www.jiumodiary.com}
  \item \url{https://www.taobao.com}
\end{itemize}

%----------------------------------------------------------------------------------------
\question{How to debug shell script?}\label{}

%----------------------------------------------------------------------------------------
\question{How to debug shell script?}\label{}

%----------------------------------------------------------------------------------------
\end{document}
